\documentclass[pdf]{beamer}
\mode<presentation>{}

\usetheme{metropolis}
\usepackage[T1]{fontenc}
\usepackage{minted}
\usepackage{graphicx}

\newcommand{\pos}{\mathbf{x}}             % position vector
\newcommand{\pvel}{\boldsymbol{\xi}}      % particle velocity
\newcommand{\mvel}{\mathbf{u}}            % macroscopic velocity
\newcommand{\colop}{\Omega}               % collision operator
\newcommand{\grad}{\boldsymbol{\nabla}}   % gradient operator
\newcommand{\mmom}{\mathbf{j}}            % macroscopic fluid momentum
\newcommand{\transM}{\mathbf{M}}          % transformation matrix
\newcommand{\relaxM}{\mathbf{S}}          % relaxation matrix

\newcommand{\expnumber}[2]{{#1}\mathrm{e}{#2}}

\newcommand{\pgrad}{\frac{\partial p}{\partial x}} % pressure gradient
\newcommand{\unorm}{\mathbf{\hat{n}}} % unit normal
\newcommand{\opidx}{{\bar{\i}}}           % opposite lattice direction
\newcommand{\fset}{\mathcal{F}}           % set of fluid cells
\newcommand{\iset}{\mathcal{I}}           % set of interface cells
\newcommand{\gset}{\mathcal{G}}           % set of gas cells
\newcommand{\domain}{\mathcal{X}}         % domain

\newcommand{\tconst}{\delta_{trans}}      % computational constant
\newcommand{\srtensor}{D_{\alpha\beta}}   % strain-rate tensor

\usepackage{beramono}
\usepackage{listings}
\usepackage{xcolor}

%%
%% Julia definition (c) 2014 Jubobs
%%
\lstdefinelanguage{Julia}%
  {morekeywords={abstract,break,case,catch,const,continue,do,else,elseif,%
      end,export,false,for,function,immutable,import,importall,if,in,%
      macro,module,otherwise,quote,return,switch,true,try,type,typealias,%
      using,while,begin},%
   sensitive=true,%
   alsoother={$},%
   morecomment=[l]\#,%
   morecomment=[n]{\#=}{=\#},%
   morestring=[s]{"}{"},%
   morestring=[m]{'}{'},%
}[keywords,comments,strings]%

\lstset{%
    language         = Julia,
    basicstyle       = \ttfamily,
    keywordstyle     = \bfseries\color{blue},
    stringstyle      = \color{magenta},
    commentstyle     = \color{ForestGreen},
    showstringspaces = false,
}


\newcommand{\pw}{0.4\textwidth}
\newcommand{\ph}{0.4\textheight}

\AtBeginSection[]
{
  \begin{frame}{Table of Contents}
    \tableofcontents[currentsection]
  \end{frame}
}

\title{An introduction to lbxflow: an open-source, computational fluid dynamics solver using the Lattice Boltzmann method}
\author{Matthew Grasinger}

\begin{document}

\begin{frame}
\titlepage
\end{frame}

\section{A first example}

\subsection{Problem description}

\begin{frame}{Physical problem}
  Poiseuille flow
  \begin{itemize}
    \item Flow through a two-dimensional channel
    \item Flow driven by a constant pressure gradient
    \item No-slip boundary conditions at walls
  \end{itemize}
\end{frame}

\begin{frame}{Physical problem}
  \begin{figure} \label{fig:poise-schematic}
    \includegraphics[width=\linewidth]{figs/poise-schematic}
    \caption{Schematic of Poiseuille flow; no-slip boundary conditions are enforced at the top and bottom boundaries, and periodic boundary conditions are enforced at the left and right boundaries.}
  \end{figure}
\end{frame}

\begin{frame}[fragile]{Input file}
  \begin{columns}
    \column{0.6\textwidth}
      \tiny
      \begin{itemize}
        \item[] <1->
      \begin{minted}[gobble=4, frame=single]{yaml}
        # simulation parameters
        nsteps: { value: 2000, expr: false}
        col_f: >
               BGK_F(init_constit_srt_const(1/6),
                     init_sukop_Fk([1e-3; 0.0]))
      \end{minted}

    \item[] <2->
      \begin{minted}[gobble=4, frame=single]{yaml}
        # lattice specifications
        ni:     { value: 20,   expr: false}
        nj:     { value: 12,   expr: false}
        dx:     { value: 1.0,  expr: false}
        dt:     { value: 1.0,  expr: false}
      \end{minted}

    \item[] <3->
      \begin{minted}[gobble=4, frame=single]{yaml}
        # material properties
        rho_0:  { value: 1.0,  expr: false}
        nu:     { value: 1/6,  expr: true}
      \end{minted}

    \item[] <4->
      \begin{minted}[gobble=4, frame=single]{yaml}
        # boundary conditions
        bcs:
          - north_bounce_back!
          - south_bounce_back!
          - periodic_east_to_west!
      \end{minted}
      \end{itemize}
    \column{0.4\textwidth}
      \small
      \begin{itemize}
        \item <1-> Time steps: 2000
        \item Pressure gradient: $\nabla p = -10^{-3}$
        \item Collision operator: BGK
        \item <2-> Domain: $L = 20, H = 12$
        \item <3-> Material properties 
          \begin{itemize}
            \item $\nu = 0.167$
            \item $\rho = 1.0$
          \end{itemize}
        \item <4-> Boundary conditions
          \begin{itemize}
            \item No-slip on north and south walls
            \item Periodic east--west
          \end{itemize}
      \end{itemize}
  \end{columns}
\end{frame}

\begin{frame}[fragile]{Input file} \label{frm:input-file-start}
      \tiny
  \begin{columns}
    \column{0.6\textwidth}
      \begin{minted}[gobble=4, frame=single,linenos]{yaml}
        # simulation parameters
        nsteps: { value: 2000, expr: false}
        col_f: >
               BGK_F(init_constit_srt_const(1/6),
                     init_sukop_Fk([1e-3; 0.0]))
        # lattice specifications
        ni:     { value: 20,   expr: false}
        nj:     { value: 12,   expr: false}
        dx:     { value: 1.0,  expr: false}
        dt:     { value: 1.0,  expr: false}
        # material properties
        rho_0:  { value: 1.0,  expr: false}
        nu:     { value: 1/6,  expr: true}
        # boundary conditions
        bcs:
          - north_bounce_back!
          - south_bounce_back!
          - periodic_east_to_west!
      \end{minted}
    \column{0.4\textwidth}
    \begin{block}{To run}
    \begin{enumerate}
     \item save the input file in a text file called \texttt{first-example.yaml} in the \texttt{lbxflow} root directory.
     \item run \texttt{julia lbxflow -f first-example.yaml} from a terminal.
     \item after a short pause, the simulation will complete and \texttt{lbxflow} will terminate.
    \end{enumerate}
    \end{block}
  \end{columns}
\end{frame}

\begin{frame}[fragile]{Callback functions}
  \begin{itemize}
    \item Input file on the previous slide will simulate the problem, but will not create any interesting output.
      \pause
    \item Need to also tell \texttt{lbxflow} what and how to process and post-process.
    \item Introduce \textit{callback functions}
      \begin{itemize}
        \item Functions of the form \texttt{f(sim, t)} where \text{sim} represents a \texttt{Simulation} object and \texttt{t} represents the current simulation time.
        \item Example:
        \begin{lstlisting}
        (sim, t) -> if (t % 100 == 0)
          println("Current time: ", t);
        end
        \end{lstlisting}
      \item Above example will print simulation time every 100 time steps.
      \end{itemize}
  \end{itemize}
  
\end{frame}

\begin{frame}[fragile]{Callback functions}
  Add the following to the input file started on slide \ref{frm:input-file-start}; it creates a comma delimited file where each subsequent row is the flow velocity profile at $x = 10$ after 100 time steps:
  \tiny
  \begin{minted}[gobble=4, frame=single]{yaml}
    # Callback Functions:
    # * callback functions are written in the julia language
    # * sim.msm.u is a 3D array of flow velocity values
    # * sim.msm.u[c, i, j] is the 'c' component of velocity at the 'i'th node in the 
    #     x-direction and 'j'th node in the y-direction.
    callbacks:
      - >
        (sim, t) -> begin
          if t % 100 == 0
            open(f -> write(f, join(vec(sim.msm.u[1, 10, :]), ',') * "\n"), 
                 "u_profiles.txt", "a");
          end
        end
  \end{minted}
\end{frame}

\begin{frame}{Callback functions}
  \begin{itemize}
    \item Callback functions can sometimes be verbose
    \item Many are provided or created by other functions
  \end{itemize}
\end{frame}

\section{Input file parameters}

\begin{frame}{Frame title}
  Watch this slide grow.
  \pause
  \begin{itemize}
    \item Hello, World!
      \pause
    \item Hello, Mars!
      \pause
    \item Hello, Alpha Centauri!
  \end{itemize}
\end{frame}

\section{Boundary conditions}

\begin{frame}{Frame title}
  \begin{columns}
    \column{0.5\textwidth}
    Left column
    \column{0.5\textwidth}
    \begin{block}{Observation 1}
      Yolo
    \end{block}
    \begin{block}{Observation 2}
      Nolo
    \end{block}
  \end{columns}
\end{frame}

\section{Collision operators}

\subsection{Constitutive relationship}
\subsection{Quasiequilibrium}
\subsection{External forcing}
\section{Callback functions}

\end{document}
