\documentclass[pdf]{beamer}
\mode<presentation>{}

\usetheme{metropolis}
\usepackage[T1]{fontenc}
\usepackage{minted}
\usepackage{graphicx}

\newcommand{\pos}{\mathbf{x}}             % position vector
\newcommand{\pvel}{\boldsymbol{\xi}}      % particle velocity
\newcommand{\mvel}{\mathbf{u}}            % macroscopic velocity
\newcommand{\colop}{\Omega}               % collision operator
\newcommand{\grad}{\boldsymbol{\nabla}}   % gradient operator
\newcommand{\mmom}{\mathbf{j}}            % macroscopic fluid momentum
\newcommand{\transM}{\mathbf{M}}          % transformation matrix
\newcommand{\relaxM}{\mathbf{S}}          % relaxation matrix

\newcommand{\expnumber}[2]{{#1}\mathrm{e}{#2}}

\newcommand{\pgrad}{\frac{\partial p}{\partial x}} % pressure gradient
\newcommand{\unorm}{\mathbf{\hat{n}}} % unit normal
\newcommand{\opidx}{{\bar{\i}}}           % opposite lattice direction
\newcommand{\fset}{\mathcal{F}}           % set of fluid cells
\newcommand{\iset}{\mathcal{I}}           % set of interface cells
\newcommand{\gset}{\mathcal{G}}           % set of gas cells
\newcommand{\domain}{\mathcal{X}}         % domain

\newcommand{\tconst}{\delta_{trans}}      % computational constant
\newcommand{\srtensor}{D_{\alpha\beta}}   % strain-rate tensor


\newcommand{\pw}{0.4\textwidth}
\newcommand{\ph}{0.4\textheight}

\AtBeginSection[]
{
  \begin{frame}{Table of Contents}
    \tableofcontents[currentsection]
  \end{frame}
}

\title{An introduction to lbxflow: an open-source, computational fluid dynamics solver using the Lattice Boltzmann method}
\author{Matthew Grasinger}

\begin{document}

\begin{frame}
\titlepage
\end{frame}

\section{A first example}

\subsection{Problem description}

\begin{frame}{Physical problem}
  Poiseuille flow
  \begin{itemize}
    \item Flow through a two-dimensional channel
    \item Flow driven by a constant pressure gradient
    \item No-slip boundary conditions at walls
  \end{itemize}
\end{frame}

\begin{frame}{Physical problem}
  \begin{figure} \label{fig:poise-schematic}
    \includegraphics[width=\linewidth]{figs/poise-schematic}
    \caption{Schematic of Poiseuille flow; no-slip boundary conditions are enforced at the top and bottom boundaries, and periodic boundary conditions are enforced at the left and right boundaries.}
  \end{figure}
\end{frame}

\begin{frame}{Physical problem}
  \begin{itemize}
    \item Domain: $L = 20, H = 12$
    \pause
    \item Pressure gradient: $\nabla p = -10^{-3}$
    \pause
    \item Material properties 
      \begin{itemize}
        \item $\nu = 0.167$
        \item $\rho = 1.0$
      \end{itemize}
  \end{itemize}
\end{frame}

\begin{frame}[fragile]{Input file}
  \tiny
  \begin{minted}[gobble=4, frame=single, linenos]{yaml}
    # simulation parameters
    nsteps: { value: 2000, expr: false}
    col_f: >
           BGK_F(init_constit_srt_const(1/6),
                 init_sukop_Fk([1e-3; 0.0]))

    # lattice specifications
    ni:     { value: 20,   expr: false}
    nj:     { value: 12,   expr: false}
    dx:     { value: 1.0,  expr: false}
    dt:     { value: 1.0,  expr: false}

    # material properties
    rho_0:  { value: 1.0,  expr: false}
    nu:     { value: 1/6,  expr: true}

    # boundary conditions
    bcs:
      - north_bounce_back!
      - south_bounce_back!
      - periodic_east_to_west!
  \end{minted}
\end{frame}

\section{Input file parameters}

\begin{frame}{Frame title}
  Watch this slide grow.
  \pause
  \begin{itemize}
    \item Hello, World!
      \pause
    \item Hello, Mars!
      \pause
    \item Hello, Alpha Centauri!
  \end{itemize}
\end{frame}

\section{Boundary conditions}

\begin{frame}{Frame title}
  \begin{columns}
    \column{0.5\textwidth}
    Left column
    \column{0.5\textwidth}
    \begin{block}{Observation 1}
      Yolo
    \end{block}
    \begin{block}{Observation 2}
      Nolo
    \end{block}
  \end{columns}
\end{frame}

\section{Collision operators}

\subsection{Constitutive relationship}
\subsection{Quasiequilibrium}
\subsection{External forcing}
\section{Callback functions}

\end{document}
